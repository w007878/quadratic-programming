\documentclass[11pt,a4paper]{article}

\usepackage{ctex}
\usepackage{amsmath}
\usepackage{amssymb}
\usepackage{geometry}
\usepackage{graphics}
\usepackage{clrscode3e}
\usepackage{listings}

\geometry{top=3cm,bottom=3cm,left=2.5cm,right=2.5cm}

\begin{document}

\title{数值最优化 期末作业}
\author{15336134 莫凡}
\maketitle
\tableofcontents
\newpage

\section{Problem 1}

\subsection{问题描述}

求解二次规划问题\[\min f(x)=(x_1-1)^2+(x_2-x_3)^2+(x_4-x_5)^2\]
\[s.t.\quad\begin{array}{rcl}
	x_1+x_2+x_3+x_4+x_5-5 & = & 0\\
	x_3-2(x_4+x_5)+3 & = & 0
\end{array}\]

初始点$x^{(0)}=(3, 5, -3, 2, -2)^T$为可行点,最优解$x^*=(1,1,1,1,1)^T$

\subsection{变量消去法}

从等式约束中可以得到\[\begin{array}{l}
x_1=-x_2-3x_4-3x_5+8\\
x_3=2x_4+2x_5-3
\end{array}\]
然后代入展开,可以得到\[f(x)=58-8x_2+2x_2^2-54x_4+2x_2x_4+14x_4^2-54x_5+2x_2x_5+24x_4x_5+14x_5^2\]
设\[t=(x_2,x_4,x_5)^T\]
则$\varphi(t)=f(x)$的梯度为
\[g(t)=\begin{bmatrix}
-8+4t_1+2t_2+2t_3\\
-54+2t_1+28t_2+24t_3\\
-54+2t_1+24t_2+28t_3
\end{bmatrix}\]
Hessian矩阵\[G=\begin{bmatrix}
4 & 2 & 2\\
2 & 28 & 24\\
2 & 24 & 28
\end{bmatrix}\]

计算G的特征值为$\{2(14+\sqrt{146}),4,2(14-\sqrt{146})\}$均大于0,所以G正定,问题是严格凸二次规划,有唯一全局最优解。

由一阶必要条件,最优解处一定有$g(t)=0$,解得$t=(1,1,1)^T$,代回解得$x=(1,1,1,1,1)^T$,是唯一全局最优解,$f(x)=0$

这个问题太简单了,所以没有代码。
\subsection{Lagrange方法}

将二次规划改写为规范形式,
\[G=2\begin{bmatrix}
1 & 0 & 0 & 0 & 0\\
0 & 1 & -1 & 0 & 0\\
0 & -1 & 1 & 0 & 0\\
0 & 0 & 0 & 1 & -1\\
0 & 0 & 0 & -1 & -1
\end{bmatrix},~
h=\begin{bmatrix}
-2\\0\\0\\0\\0
\end{bmatrix},~
A=\begin{bmatrix}
1 & 0\\
1 & 0\\
1 & 1\\
1 & -2\\
1 &-2
\end{bmatrix},~
b=\begin{bmatrix}
5\\-3
\end{bmatrix}\]

后面采用零空间方法求解方向

\begin{lstlisting}[MATLAB]
% 原问题
G = 2*[1 0 0 0 0; 0 1 -1 0 0; 0 -1 1 0 0; 0 0 0 1 -1; 0 0 0 -1 1];
h = [-2;0;0;0;0];
A = [1 0; 1 0; 1 1; 1 -2; 1 -2];
b = [5;-3];

% 对A进行QR分解
[Q, R] = qr(A);
Q1 = Q(1:5, 1:2);
Q2 = Q(1:5, 3:5);
R1 = R(1:2, 1:2);

x0 = [3; 5; -3; 2; -2];
Z = Q2;
A_ = Q1 * inv(R1)';

d = (Z' * G * Z) \ (-Z' * (h + G * x0));

x = x0 + Z * d;
disp('x=')
disp(x);
\end{lstlisting}
\section{Problem 2}

\subsection{问题描述}

\[\min f(x)=9-8x_1-6x_2-4x_3+2x_1^2+2x_2^2+x_3^2+2x_1x_2+2x_1x_3\]
\[s.t.\quad\begin{array}{rcl}
2-x_1-x_2-2x_3 & \ge & 0\\
x_i & \ge & 0,~~i=1,2,3
\end{array}\]

初始点$x^{(0)}=(0.5,0.5,0.5)^T$为可行点,最优解为$x^*=(\frac{4}{3},\frac{7}{9},\frac{4}{9})^T$

\subsection{起作用集方法}

\begin{codebox}
	\Procname{Active-Set($G,h,x^{(0)}$)}
\end{codebox}

\end{document}